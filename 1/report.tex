\documentclass[a4paper,article,10pt,notitlepage,onecolumn,oneside]{memoir}

%%%%%%%%%%%%%%%%
%%% PACKAGES %%%
%%%%%%%%%%%%%%%%

%language
\usepackage[english]{babel}
\usepackage[english]{isodate}

%fonts
\usepackage{fontspec}

%graphics
\usepackage{placeins}
\usepackage{graphicx}
\usepackage{tikz}
\usepackage{tikz-cd}
\usepackage{xcolor}
\usepackage{lipsum}

%symbols
\usepackage{amsmath}
\usepackage{amssymb}
\usepackage{bm}
\usepackage{listings}

%bibtex
\usepackage[backend=biber]{biblatex}
\addbibresource{ass1.bib}

%last because technical reasons
\usepackage{hyperref}
\usepackage{memhfixc}

%%%%%%%%%%%%%
%%% SETUP %%%
%%%%%%%%%%%%%

%narrow margins to capitalize on page real estate
\setlrmarginsandblock{2cm}{2cm}{1}
\setulmarginsandblock{2cm}{4cm}{1}
\setcolsepandrule{1cm}{0pt}
\checkandfixthelayout 

%nice fonts
\setmainfont{Linux Libertine O} %download: http://libertine-fonts.org/
\setsansfont{Linux Biolinum O} %download: http://libertine-fonts.org/
\setmonofont[Scale=0.9]{Fira Mono} %download: https://github.com/mozilla/Fira

\makeatletter
\newcommand{\verbatimfont}[1]{\def\verbatim@font{#1}}%
\makeatother
\verbatimfont{\ttfamily}

%hyperlink appearance
\hypersetup{
  colorlinks,
  linkcolor={red!50!black},
  citecolor={blue!50!black},
  urlcolor={blue!80!black}
}

%code listings
\lstset{
  language=Haskell, %change later?
  basicstyle=\footnotesize\ttfamily,
  numbers=left,
  numbersep=10pt,
  tabsize=2,
  showstringspaces=false, 
  keywordstyle=\color{blue},
  stringstyle=\color{orange!50!black},
  emphstyle=\color{green!50!black}
}

%iso 8601 date format using EN-dashes
\isodash{--}\isodate

%nice headers and footers
\setlength{\headheight}{45pt}
\nouppercaseheads
\makepagestyle{fancy}
\makeoddhead{fancy}{}{}{\begin{minipage}{0.35\textwidth}\thetitle\\\theauthor\end{minipage}}
\makeoddfoot{fancy}{\thedate}{}{\thepage\ of \thelastpage}
\pagestyle{fancy}

%pagestyles: book chapter cleared epigraph title part titlingpage
\copypagestyle{chapter}{fancy}
\copypagestyle{title}{fancy}

%no section numbering
\setsecnumdepth{chapter}

%spacing
\setlength\parindent{0pt}
\setlength\parskip{0.5em}
%\titlespacing*{\section}{0pt}{1\baselineskip}{1\baselineskip}
%\titlespacing*{\subsection}{0pt}{0.5\baselineskip}{0.5\baselineskip}
%\titlespacing*{\subsubsection}{0pt}{0.25\baselineskip}{0.25\baselineskip}

\date{\protect\today}
\title{Subscript Interpreter \\ Assignment 1, AP 2018 CSICU}
\author{Karl D. Asmussen \texttt{twk181}}
\newcommand\thecourse{Advanced Programming 2018\\CPH Univ. CS Dept.}

\begin{document}
\mainmatter

\maketitle

\section{The \lstinline{SubsM} Monad}

\newcommand\thecodefile{SubsInterpreter.hs}
\newcommand\thecodedir{handin/src/}
\newcommand\showlines[2]{%
\lstinputlisting[firstnumber=#1, firstline=#1, lastline=#2, caption={\thecodefile}]{\thecodedir\thecodefile}%
}

The subscript interpreting monad \lstinline{SubsM} is a Reader-State combo.
As such, I have added the \verb+mtl+ to dependencies and implemented the appropriate
type classes for ease of use.

\showlines{93}{106}

This in turn makes it easy to define a number of utility functions, such as one
to drop a defined variable from the environment.

I've switched to a new error type, for ease of testing failure behavior:

\showlines{56}{68}

\section{Evaluation}

Evaluation of the AST is implemented in straightforward fashion with some
stylistic preferences for terseness and the use of combinators over \lstinline{do}-notation.

\showlines{157}{158}
\showlines{163}{164}

The most interesting part is the evaluator for array comprehensions (\lstinline{evalCompr}, l. 172),
which cleverly avoids clobbering variables using a novel combinator.

\showlines{308}{310}

\section{Testing}

I've implemented a few basic tests that make use of the fact that \lstinline{Tasty}'s assertions
live in the \lstinline{IO} monad, in order to generate random test cases.

\renewcommand\thecodefile{Test.hs}
\renewcommand\thecodedir{handin/tests/}

\showlines{23}{27}

I've also used my new error type to test for division by zero, which it turned out I had
forgotten to implement.

\showlines{34}{39}

\appendix

\section{Code Listing}

\lstinputlisting[caption={SubsInterpreter.hs}]{handin/src/SubsInterpreter.hs}
\lstinputlisting[caption={Main.hs}]{handin/src/Main.hs}
\lstinputlisting[caption={Tests.hs}]{handin/tests/Test.hs}

%\tableofcontents
%\twocolumn

%\appendix
 
%\backmatter

%\printbibliography%
%\listoffigures
%\listoftables
%\printindex%

\end{document}
