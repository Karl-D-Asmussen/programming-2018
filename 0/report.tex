\documentclass[a4paper,article,oneside,10pt,twocolumn]{memoir}

\usepackage[english]{babel}
\usepackage[english]{isodate}

\usepackage{listings}
\usepackage{ulem}
\usepackage{fontspec}
\usepackage{xcolor}

\setmainfont{Linux Libertine O}
\setsansfont{Linux Biolinum O}
\setmonofont[Scale=0.9]{Fira Mono}

\makeatletter
\newcommand{\verbatimfont}[1]{\def\verbatim@font{#1}}%
\makeatother
\verbatimfont{\ttfamily}

\setsecnumdepth{chapter}

\lstset{
  language=Haskell,
  basicstyle=\footnotesize\ttfamily,
  numbers=left,
  numbersep=5pt,
  tabsize=2,
  showstringspaces=false, 
  keywordstyle=\color{blue},
  stringstyle=\color{orange!50!black},
  emphstyle=\color{green!50!black}
}

\setlrmarginsandblock{2cm}{2cm}{1}
\setulmarginsandblock{2cm}{4cm}{1}
\setcolsepandrule{1cm}{0pt}
\checkandfixthelayout 

\setlength{\headheight}{45pt}
\nouppercaseheads
\makepagestyle{fancy}
\makeoddhead{fancy}{}{}{\begin{minipage}{0.25\textwidth}\thetitle\\\theauthor\\\thecourse\end{minipage}}
\makeoddfoot{fancy}{\thedate}{}{\thepage\ of \thelastpage}
\pagestyle{fancy}

\setlength{\parindent}{0pt}
\setlength{\parskip}{0.5em}

\newcommand\thecourse{Advanced Programming 2018\\CPH Univ. CS Dept.}

\title{Assignment 0}
\author{Karl D. Asmussen \texttt{twk181}}
\isodash{--}\isodate
\date{\protect\today}

\begin{document}
\maketitle\thispagestyle{fancy}


\section{Showing}

My implementation of \lstinline{showExp} uses the \lstinline{ShowS} type internally,
because I personally abhor writing a bunch of \lstinline{++}'s. I use a number of helper functions
that eliminate some boilerplate, and yet end up writing the same line five times. All tests pass.

My implementation of \lstinline{showCompact} uses a precedence system internally
to get everything \emph{mostly} right, accounting for parentheses levels with precedence levels
and for left/right association by offsetting the precedence levels in either branch. I've
handcrafted the case for prefix negation, and used an entirely different strategy for
avoiding boilerplate. All tests pass.

\section{Utility}

My implementation of \lstinline{extendInv} crucially \lstinline{seq}'s the
extant environment with the newly constructed one. This is to allow the use
of \lstinline{undefined} to disallow variable use and definition, as any attempt
to even construct an environment would be met with runtime errors. All tests pass.

\section{Eval}

My implementation of \lstinline{evalSimple} actually defers directly
to \lstinline{evalFull} using the above mechanism to ensure error compliance. There
are no tests to check for this. All tests pass.

My implementation of \lstinline{evalFull} also defers, this time to
\lstinline{evalErr} and simply converts \lstinline{Left} errors into runtime
exceptions using my utility function \lstinline{showErr}.
All tests pass.

\textit{A/N: \lstinline{showErr} doesn't use \lstinline{ShowS}, and it shows.}

My implementation of \lstinline{evalErr} is an alias of \lstinline{evalEager}.
All tests pass.

My implementation of \lstinline{evalEager} is really the meat of this
entire assignment. It uses a helper function called \lstinline{evalM} which is based
on \lstinline{MonadReader} and \lstinline{MonadError}, generic over implementation. In
the specific instance of \lstinline{evalEager} the implementation becomes
\lstinline{ReaderT Env (Either ArithError) Integer}.

The choice of monad transformers to solve this assignment is massively overkill, I admit.
On the other hand they are also the right tool for the job. The archetype of Reader is scoped
variables, with the \lstinline{MonadReader} implementation calling the primitives \lstinline{local}
and \lstinline{ask}. Error-handling is neatly done with \lstinline{Either}, but part of
me wishes there was a neat implementation of continuation-based error handling in monad form
in the mtl. All tests pass.

My implementation of \lstinline{evalLazy} is far from stellar. Its eager
cousin gets to have all the monad transforming fun, but unfortunately also uses monadic
sequencing to force let-bindings in the interpreted language.

I have so far failed to find a way to defer error-reporting in a general sense, but I can
still cheat your test suite. All tests pass.

\section{Quality and Motivation}

Code quality is probably below par. I have been coding mostly straightforward, but
especially in the \lstinline{show*} family, my handling of boilerplate varies; I should have
worked more on eliminating duplicate code from both, or implementing the simpler in terms
of the more complex with a flag of some sort.

\section{Style}

I like to use do-notation in this manner:

\begin{lstlisting}
myFun a b c = do a' <- a
                 b' <- b
                 c' <- c
                 return (a' + b' + c')
\end{lstlisting}

Hlint can \sout{suck my d} sue me.

Also can we start allowing lines over 80 characters long?
The vt100 is \emph{forty years old now.}

\end{document}
